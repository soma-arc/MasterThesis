%#!uplatex main.tex

\section{Application}

In this section, we introduce advanced usage of IIS.

\subsection{Render internal area}

%% 二次元のフラクタルの内側だけを描く.全ての円盤が接していれば極限集合
%% で二分割することができる

\subsection{Render Edge of the circles}

%% ヤコビアンを累積させていき,最後の円周からの距離を累積させたヤコビア
%% ンで割ることで,円周から点までの距離を得ることができる.

\subsection{Geometrical Representation of M\"obius Transformation Groups}

%% メビウス変換を円や球の反転で定義することによって,より直観的に生成元
%% を得て描画することができる.

\subsection{Sphairahedra and Three-dimensional Fractals}

%% 球面体のタイリングもIISで描画することができる