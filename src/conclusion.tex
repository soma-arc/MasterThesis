%#!uplatex main.tex

\section{Conclusion}

In this paper, we introduced an efficient algorithm
called Iterated Inversion System (IIS) to visualize Kleinian
groups based on circle inversions and sphere inversions.

IIS has a constraint that we have to compute in parallel;
% 和訳はできますけど、言いたいことがちょっとわかりません。
nevertheless it is useful. The application range of IIS is broad.
We can render many images related to Kleinian groups and tiling using
IIS.

There is a fractal called \textit{pseudo-kleinian}, and
% IISが発表される前から、擬クラインと呼ばれる「フラクタルの描画アルゴリズム」があって、
there are algorithms related to pseudo-kleinian.
% それに関連する描画のテクニックやバリエーションもいくつか知られていた。
The algorithms are fast like IIS, but they are not a Kleinian group which
mathematician means.
% 擬クラインは3次元フラクタル図形を高速に描画するという意味ではIISに似ており、
% メビウス変換が実際に使われている部分もあり、関連はあると思われる。
% しかし、擬クラインは数学者が考えているクライン群とは異なるものである。
However, IIS and mathematics are developed close
together.
% Howeverはいらないかも
Therefore IIS is taken notice by researchers who study Kleinian groups as
pure mathematics and computer graphics.

On the other hand, there are also Kleinian groups, which we can not
% Keleinian groups の前に a certain family of を入れる。
visualize using IIS.
Our final goal is that for all Kleinian groups, we develop this kind of
efficient algorithm and get mathematical results from obtained images.
