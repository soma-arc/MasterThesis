
%#!uplatex main.tex

\section{Introduction}

Kleinian group theory is one of the fields of mathematics studying 
M\"obius transformation groups.
It is well suited to visualization and experiment;
Actually, Felix Klein, a mathematician who study Kleinian groups theory
in 19 century and his students also leave exquisite visualized images
without computer.
There are aspects that mathematicians advance research from
visualization and experiment.

After a computer appeared, various visualization and calculation are
performed by computer.
Visualized image of Kleinian groups often have beautiful fractal shape.
Thus, there are people who enjoy rendered images as an art.
For example,
\textit{Fractalforums}\footnote{\url{https://fractalforums.org/}}
community gathers many fractal enthusiasts and discusses fractals.
The fractals generated by Kleinian group theory are also came up for
discussion.
In a sense, Kleinian group theory is interdisciplinary area between
mathematics and arts.

Mumford, Series, and Write wrote a book called
\textit{Indra's Pearls} \cite{MumfordSeriesWright200204}.
The book is written for non-mathematician, and contains explaination
about Kleinian group theory, many beautiful visualized images, and
methods of visualization.
Thus, not only math enthusiast but also programmers enjoy the book.

However, the book deal with a small part of Kleinian group.
Because of the properties of the group, we can not
visualize all of the types of Kleinian groups in real time
inspite of the growth of the performance of personal computer.

Our goals are to visualize all of the Kleinian groups in real time by
personal computer and help us understand the Kleinian group theory
intuitively with compute softwares.

As the first step to the goals, we invent an algorithm called
\textit{Iterated Inversion System (IIS)}.
IIS is an algorithm to render Kleinian group based on circle or sphere
inversions.
It visualize not only two dimensional Kleinian group but also three
dimensional Kleinian group.
In this paper, we introduce basic usage of the IIS algorithm and its
applications.
