%#!uplatex main.tex

\section{Introduction}

Kleinian group theory is one of the fields of mathematics studying 
M\"obius transformation groups.
Kleinian group theory is advanced by mathematicians in the nineteenth
century.
Felix Klein and his student, Robert Fricke studied M\"obius
transformation groups.
Henri Poincare named such groups \textit{Kleinian Groups}.
Moreover, composing Geometry, Algebra, and Analysis,
Poincare built foundation of Kleinian groups theory.
Klein and Poincare were rivals to study Kleinian group theory.

%ポアンカレはクライン群の研究を双曲幾何学の問題としてとらえる方法を提案した。20世紀後半にサーストンの功績により、双曲幾何学が現代数学の一分野として再認識されるようになると、クライン群も現代幾何学の観点から研究されるようになり、クライン群のもつ複雑さが、現代数学の手法によって少しずつ解明されるようになり現在に至っている。

M\"obius transformation group is well suited to visualization and
experiment; Actually, Klein and his students also leave beautiful
visualized images of a Kleinian group without a computer.
There are aspects that mathematicians advance research from
visualization and experiment.

After a computer appeared, various visualization and calculation are
performed by computer.
Visualized images of Kleinian groups often have beautiful fractal shape.
Thus, some people enjoy rendered images as arts.
% some people や enjoy がややぼんやりとしている。
% 「CGの愛好家」が、「フラクタルのもつ複雑な図柄に魅惑された」など、具体的に書くとよい。
For example,
\textit{Fractalforums}\footnote{\url{https://fractalforums.org/}}
community gathers many fractal enthusiasts and discusses fractals.
The fractals generated by Kleinian group theory also come up for
discussion.
In a sense, Kleinian group theory is an interdisciplinary area between
mathematics and arts.

Mumford, Series, and Write wrote a book called
\textit{Indra's Pearls} \cite{MumfordSeriesWright200204}.
The book is written for non-mathematician and contains explanation
about Kleinian group theory, many beautiful visualized images, and
methods of visualization.
% The book -> This book
% explanation -> (countable)-> explanations
Thus, not only math enthusiast but also programmers enjoy the book.
% math maniacs

However, the book deal with a small part of Kleinian group.
Because of the properties of the group, we cannot
visualize all of the types of Kleinian groups in real time
in spite of the growth of the performance of a personal computer.
% only a small part
% properties of the group を何か具体的に
% ~できるわけではない。→困難さがある。

Our goals are to visualize all of the Kleinian groups in real time by
personal computer and help us understand the Kleinian group theory
intuitively with software.
% Our finial goal 

As the first step to the goals, we invent an algorithm called
\textit{Iterated Inversion System (IIS)}.
IIS is an algorithm to render Kleinian group based on circle or sphere
inversions.
It visualizes not only two-dimensional Kleinian group but also
three-dimensional Kleinian group.
%two -> three, three-> four
In this paper, we introduce the basic usage of the IIS algorithm and its
applications.

