%#BIBTEX biber --bblencoding=utf8 -u -U --output_safechars main
%#!uplatex main.tex

\section{Preparation}

In this section, we introduce some mathematical terms and prerequisites
to understand the IIS algorithm and basic usage of IIS.

\subsection{Transformation Group Theory}

In this paper, we use terms about Kleinian groups used in Indra's
Pearls \cite{MumfordSeriesWright200204}.
The word \textit{group} represents an algebraic group which is
central concept about group theory.
The algebraic group is a set which has multiplication, satisfies the
associative law, has a unit, and has an inverse element for each
element.

Also, a transformation group is an algebraic group consists of transformations 
on a topological set $X$.  Here a transformation is a homeomorphism
$f:X \to X$.

We will introduce the M\"obius transformation group and Kleinian group
as a subgroup of the M\"obius transformation group. 
In this paper, we mainly consider the cases $X=\tilde{\mathbb{C}}$,
which is the union of the complex set and the infinity, 
and $X=\tilde{ \mathbb{R} }^3$, which is the union of $\mathbb{R}^3$ and the infinity.
In our context, we suppose proper topologies on $\tilde{\mathbb{C}}$ 
by the complex topology of $P^1(\mathbb{C})$ and on $\tilde{ \mathbb{R} }^3$
by one-point compactification from $\mathbb{R}^3$ respectively.

We will introduce M\"obius transformation group on $\tilde{\mathbb{C}}$ and $\tilde{ \mathbb{R} }^3$.  
First, the two-dimensional M\"obius transformation group
\textrm{M\"ob}$(\tilde{\mathbb{C}})$ is the set of linear fractional transformations $f(z)=\frac{az+b}{cz+d}$, 
where $a,b,c,d$ are complex numbers and they satisfy $ad-bc = 1$. Here we naturally extend a complex function $f(z)$ 
into a transformation on $\tilde{\mathbb{C}}$.  
A linear fractional transformation is a conformal orientation-preserving homeomorphic map.
See also section 2.3.
From another point of view, a M\"obius transformation is the composition
of an even number of circle inversions.  
About inversions, see section 2.2.
In the same manner, we consider the three-dimensional M\"obius
transformation group \textrm{M\"ob}$(\tilde{\mathbb{R}}^3)$
In the three-dimensional case, we define a M\"obius transformation as
the composition of an even number of sphere inversions.

\subsection{Inversions in Circles or Spheres}

In the two-dimensional cases, it is well known that M\"obius transformations on $\tilde{\mathbb{C}}$
are composed of an even number of circle-inversions and line-symmetries.
Here, if a circle is centered at $C\in\mathbb{C}$ and it has a radius $R\in\mathbb{R}$ ($R>0$).
Then the formula of the inversion along the circle is given by
$f(z) = \frac{R^2}{~\overline{z -C}~} + C$.
Since $f \circ f$ is the identity map, the inversion map is a homeomorphism on $\tilde{\mathbb{C}}$.
Remark that the inversion is an orientation-reversing conformal map on $\tilde{\mathbb{C}}$.

We often distinguish the inversion from a line symmetry map.
But in this context, by interpreting the line on the complex plane as a circle centered at the infinity, 
we may regard the line symmetry as a kind of the inversion. 
In the sequel, we will call inversion both of circle-inversions and line-symmetries.
In this category, any inversion is orientation-reversing, conformal,
and homeomorphic on $\tilde{\mathbb{C}}$.

Here we will make a short note on the Jacobian of inversion.
If the inversion is a line symmetry, the determinant of the Jacobian is $-1$.
Otherwise, if the inversion is a circle-inversion, remarking that a circle inversion
preserves any angles, we obtain that Jacobian mapping is given by the composition of multiplication by a complex constant
and complex conjugate.
Thus the determinant of the Jacobian at a point $P\in\mathbb{C}$ is given by the following formula:
\[ \mathrm{Det}(\mathrm{Jacobian}) = R^4 / distance(P,~C)^4 \]%
Here the Jacobian is a composition of a scaling and a rotation and a complex conjugate, and we obtain the following.
\[ \mathrm{scaling~factor~at~} P  = R^2 / distance(P,~C)^2. \]

In a similar way as the definition of a circle-inversion,
a sphere-inversion can be defined as follows.
Fix a sphere $C$ in $\mathbb{R}^3$.  Let its center be $C$ and its radius be $R$. 
A sphere-inversion $I_C$ is a homeomorphism on $\tilde{\mathbb{R}}^3$ 
determined by a map from a point $P$ to a point $Q = I_C(P)$ such that (i) $Q$ is on the half line $CP$ and 
that (ii) $\overline{CP}\cdot \overline{CQ} = R^2$.
As above, the definition of a circle-inversion can be extended to a plane-symmetry
in the same way as a line-symmetry in the two-dimensional case.

\subsection{M\"obius Transformations and the Hyperbolic Geometry}

First, we review the definition of a group action. 
Let $G$ be a group and $X$ be a set. A map $G \times X \to X : (g,x) \mapsto g\cdot x$
is called a $G$-action on $X$ if it satisfies the following conditions.
\par \qquad (1) $e \cdot x = x$ for all $x \in X$. (Here, $e$ denotes the unit of $G$.)
\par \qquad (2) $(gh) \cdot x = g \cdot (h\cdot x)$ for all $g, h \in G$ and for all $x \in X$.

In this study, we handle $PSL_2\mathbb{C}$-action on $\hat{\mathbb{C}}$.
Here $PSL_2\mathbb{C}$ is the projective space of $2 \times 2$ complex matrices $A$ with $\mathrm{det}(M)=1$.
M\"obius transformation is defined as a linear fractional transformation
$f(z)=\dfrac{az+b}{cz+d}$ for complex variable $z$ where constants
$a, b, c, d$ are complex numbers and satisfy $ad - bc = 1$.

If we consider a map $\varphi: \dfrac{az+b}{cz+d} \mapsto \begin{pmatrix}a & b \\ c& d \end{pmatrix}$,
from the set of linear fractional transformations to $PSL_2\mathbb{C}$.
This map $\varphi$ is a group isomorphism, and it gives an induced action of $PSL_2\mathbb{C}$ on $\hat{\mathbb{C}}$.
In the sequel, we regard the action of linear fractional transformations
and that of $PSL_2\mathbb{C}$ without distinction.
For more details about M\"obius transformation, refer
\cite{MumfordSeriesWright200204}\cite{marden_2016}.

M\"obius transformation is deeply related to the hyperbolic geometry.
When all of the coefficients in $f(z) = \dfrac{az + b}{cz + d}$
are real number, $f(z)$ preserve upper-half plane.
Moreover, $f(z)$ preserves the hyperbolic metric $\dfrac{dz^2}{(\mathrm{Im}z)}$
on the upper-half plane.
Such linear fractional transformations correspond to an element of $PSL_2\mathbb{R}$.
Thus the $PSL_2\mathbb{R}$-action on the upper-half plane is
the isometric transformation group of the hyperbolic plane $\mathbb{H}^2$.

When the coefficients $a, b, c, d$ are complex numbers, 
the action can be thought as an isometric transformation group on the hyperbolic space $\mathbb{H}^3$.
We consider the uppe-half model of the hyperbolic space $\mathbb{H}^3$.
In fact, 
\[\mathbb{H}^3 = \{ (z,t) \in \mathbb{C}\times \mathbb{R}\mid t>0 \} \cup \{\infty\} \]
is the upper-half model with the hyperbolic metric $\dfrac{dz^2+dt^2}{t^2}$ and 
$\hat{\mathbb{C}} = \{ (z,0) \mid z \in \mathbb{C}\} \cup \{ \infty\}$
is the set of infinity. The two-dimensional M\"obius transformation group acts on $\hat{\mathbb{C}}$
naturally, and it is well known that this action can be extended to 
an action on the upper-half model as isometric transformations.
This extension is called \textit{\lq Poincar\'e extension\rq}.
As above, researching structures of the M\"obius transformation groups is heavily
related to studying three-dimensional hyperbolic geometry and hence hyperbolic manifolds.
For more details, refer 
\cite{Marden200705outerCircles}\cite{taniguchi_okumura199610invitation}.

In this paper, we also consider the three-dimensional M\"obius
transformation on $\hat{\mathbb{R}^3} = \mathbb{R}^3\cup\{\infty\}$.
The definition of the three-dimensional M\"obius transformation in the two ways as follows.
One is the simple extension of the two-dimensional version.
A M\"obius transformation is defined as a composition of an even number of sphere-inversions 
and plane-symmetry.
The other is the representation of the isometry group of the four-dimensional hyperbolic space.
To represent three-dimensional M\"obius transformation using matrix,
we consider a quaternion $2 \times 2$ matrix group called
$Sp^k(1,1)$. About this topic, refer
\cite{sakugawa2010limit}\cite{sakugawa2007master}.

Three-dimensional M\"obius transformations are deeply related to
the four-dimensional hyperbolic space. In fact, the upper-half space model of
the four-dimensional hyperbolic space $\mathbb{H}$ is homeomorphic to the 
four-dimensional open ball and its boundary (the infinity point set of the hyperbolic
space) is $\hat{\mathbb{R}^3} = \mathbb{R}^3\cup\{\infty\}$.
Using Poincar\'e extension, a sphere-inversion in $\hat{\mathbb{R}^3}$
is extended to an (orientation-reversing) isometric transformation of $\mathbb{H}^4$.
Thus the composition of an even number of sphere-inversions corresponds to an orientation-preserving
isometry, and it is known that any isometry of $\mathbb{H}^4$ is given in this way.
In this point of view, three-dimensional M\"obius transformations are
deeply related to the four-dimensional hyperbolic space and four-dimensional hyperbolic manifolds.

\subsection{Classification of M\"obius Transformations}

Excluding the identical map, we will classify M\"obius transformations
$f(z) = \dfrac{az + b}{cz + d}$ into three types.
The types are \textit{elliptic}, \textit{parabolic}, and \textit{loxodromic}.
It is well-known that the standard form of $PSL_2\mathbb{C}$
is either one of $\begin{pmatrix}1 & \lambda \\ 0 & 1 \end{pmatrix}$(, $\lambda$ is a non-zero constant,) or
$\begin{pmatrix}\lambda & 0 \\ 0 & \lambda^{-1} \end{pmatrix}$(, $\lambda$ is a non-zero constant,) 
from a famous theorem in linear algebra.
If $X=\begin{pmatrix}a & b \\ c & d \end{pmatrix}$ is conjugate to the former one,
then $f(z)$ is called a parabolic transformation.
If $f(z)$ is parabolic, the number of fixed point is one, and
the matrix $X$ satisfy $\mathrm{tr}^2X = 4$.

When the standard form of $X$ is
$\begin{pmatrix}\lambda & 0 \\ 0 & \lambda^{-1} \end{pmatrix}$,
we have the following two cases. 
One is the case $|\lambda|=1$ and the other is the case $|\lambda|\neq 1$ 
If $|\lambda|=1$ then $f(z)$ is called an elliptic transformation.
When $f(z)$ is elliptic,  the fixed point set of $f(z)$ consists of two distinct points,
and $f(z)$ gives a rotation around these fixed points.
And $f(z)$ is elliptic if and only if $\mathrm{tr}^2X > 4$.
Remark that if the order of a M\"obius transformation $f(z)$ is finite
then it is elliptic.

If the standard form is
$\begin{pmatrix}\lambda & 0 \\ 0 & \lambda^{-1} \end{pmatrix}$
with $|\lambda| \neq 1$, then $f(z)$ is called a loxodromic transformation.
In a special case, if $\lambda$ is a real number other than $\pm 1$, the transformation is called
a \textit{hyperbolic} transformation too.
When $f(z)$ is loxodromic the number of the fixed point set of $f(z)$ is two.
One of the fixed points is called an attracting fixed point and the other is called a 
repelling fixed point. This is because the orbit space of $f(z)$ of a
general point $z_0$ has two limit point and one of two is the limit of
$f^n(z_0)$ for $n\to \infty$, and the other is the limit of $f^n(z_0)$
for $n\to -\infty$. 
Also, $f(z)$ is loxodromic if and only if $\mathrm{tr}^2X < 4$.

The set of three-dimensional M\"obius transformations also have a similar
classification into six types.
In three-dimensional case, each type \lq parabolic,\rq \lq elliptic,\rq \lq loxodromic\rq has 
a modifier \textit{simple} and \textit{compound}.
Thus there are six variations of classification types, for example, \lq\lq simple elliptic\rq\rq or
\lq\lq compound loxodromic.\rq\rq
For detailed information of classification of three-dimensional M\"obius transformation, see chapter 4.3.2.

\subsection{Kleinian Groups}

A Kleinian group is a group originated from the name of Felix Klein who is
a famous mathematician in the nineteenth century.
A group $G$ is a Kleinian group if it satisfies the following two
conditions. 
First, $G$ is a subgroup of the M\"obius transformation group.
Second, the $G$-action on $\mathbb{H}^3$,
is properly discontinuous.
Here a $G$-action on $\mathbb{H}^3$
is properly discontinuous if and only if
for any compact set $K \subset \mathbb{H}^3$
there are an only finite number of elements $g \in G$ such that
$\gamma (K) \cap K \neq \O$.

When the $G$-action is properly discontinuous, 
the $\epsilon$ neighborhoods of points in the orbit space are disjoint
for a positive $\epsilon$.
This follows that there exists a
fundamental domain with positive volume in $\mathbb{H}^3$.
Because a M\"obius transformation gives an isometric transformation of
$\mathbb{H}^3$, a Kleinian group gives a hyperbolic three-manifold 
as a quotient space of the group. 
Basic properties of a Kleinian group are described in chapter 2 of \cite{marden_2016}.

In this paper, we consider a M\"obius transformations in a broad sense, that is,
a map which is the composite of
any number of inversions.
In order to distinguish an original M\"obius transformation,
we call it \textit{an extended M\"obius transformation}.
The composite of odd numbers of inversions
is an orientation-reversing map represented by $f(z)=\dfrac{a{\bar{z}}+b}{c{\bar{z}}+d}$, where
$a, b, c, d$ are complex numbers and $ad-bc = 1$.

An extended M\"obius transformation induces an isometric mapping 
(not necessarily to be orientation-preserving) on the three-dimensional hyperbolic space $\mathbb{H}^3$ by
Poincar\'e expansion.
We call a properly discontinuous subgroup of the extended M\"obius transformation group
\textit{an extended Kleinian group}.

\subsection{Limit Set}

Let $G$ be a Kleinian group.
For a general point $z_0 \in \hat{\mathbb{C}}$,
a point $p  \in \hat{\mathbb{C}}$ is a limit point 
of the orbit space of $z_0$ if and only if 
there exists a sequence $\{ g_i \} \subset G$ such that 
$\displaystyle \lim_{i\to\infty} g_i(z_0) = p$.
We call the closure of all limit points of an orbit space $Gz_0$ \textit{the limit set} of $G$.
The limit set is denoted by $\Lambda(G)$.
Some of the properties are very useful to visualize the limit set of a Kleinian group.
\begin{itemize}
\item The limit set does not depend on a general point $z_0$.  
That is, the closure of all limit points of any orbit space is $\Lambda(G)$.
\item The limit set is preserved by $G$-action.  That is,
the image of action $g\in G$ of a point $p\in \Lambda(G)$  
      also is included in the limit set $\Lambda(G)$.
\item For any element $g \in G$, the fixed point of $g$ is contained in the limit set.
\end{itemize}
Much more properties of the limit set is also described in chapter 2.4.1 of
\cite{marden_2016}.

Here we have some concepts to describe limit points of an orbit. 
One is an algebraic limit and the other is a geometric limit.
When we fix a set of generators of a Kleinian group $G$, we represent any elements in $G$ 
as a word of these generators and their inverses.
Let a loxodromic element $g$ be a word $a_1a_2\cdots a_r$.
Suppose that a point sequence $\{ (a_1a_2\cdots a_r)^n\cdot z_0 \}$ converges to a point $p$ in the limit set.
This limit point $p$ is one of the fixed points of $g$.
Thus we consider a formula $p = (\overline{a_1a_2\cdots a_r})z_0$, where $\overline{a_1a_2\cdots a_r}$
is a circulating decimal of generators (and their inverses.)
However $\overline{a_1a_2\cdots a_r}$ is not an element in $G$, supposing an abstract infinite word,
we may consider a point on the limit set as a point in an orbit.
This way of thinking is called an algebraic limit of $G$.
On the other hand, the original definition of a limit point is obtained in a geometrical way of thinking.
Thus such limit point is also called a geometric limit.
