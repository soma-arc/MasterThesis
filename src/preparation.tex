%#BIBTEX biber --bblencoding=utf8 -u -U --output_safechars main
%#!uplatex main.tex

\section{Preparation}

In this section, we introduce some mathematical terms and prerequisites
to understand the IIS algorithm and basic usage of IIS.

\subsection{Transformation Group Theory}

In this paper, we use terms about Kleinian groups used in Indra's
Pearls \cite{MumfordSeriesWright200204}.
The word \textit{group} represents an algebraic group which is
central concept about group theory.
Algebraic group is a set which has multiplication, satysfies the
associative law, has a unit, and has an inverse element for each
element.

Also, a transformation group is an algebraic group consists of transformations 
on a topological set $X$.  Here a transformation is a homeomorphism
$f:X \to X$.

We will introduce the M\"obius transformation group and Kleinean group
as a subgroup of the M\"obius transformation group. 
In this paper, we mainly consider the cases $X=\tilde{\mathbb{C}}$,
which is the union of the complex set and the infinity, 
and $X=\tilde{ \mathbb{R} }^3$, which is the union of $\mathbb{R}^3$ and the infinity.
In our context we suppose proper topologies on $\tilde{\mathbb{C}}$ 
by the complex topology of $P^1(\mathbb{C})$ and on $\tilde{ \mathbb{R} }^3$
by one-point compactification from $\mathbb{R}^3$ respectively.

We will introduce M\"obius transofmation group on $\tilde{\mathbb{C}}$ and $\tilde{ \mathbb{R} }^3$.  
First, the two-dimensional M\"obius transformation group
\textrm{M\"ob}$(\tilde{\mathbb{C}})$ is the set of linear fractional transformations $f(z)=\frac{az+b}{cz+d}$, 
where $a,b,c,d$ are complex numbers and they satisfy $ad-bc = 1$. Here we naturally extend a complex function $f(z)$ 
into a transformation on $\tilde{\mathbb{C}}$.  
A linear fractional transformation is a conformal orientation-preserving homeomorphic map.
See also section 2.3.
From another point of view, a M\"obius transformation is the composition of even number of circle inversions.  
About inversions, see section 2.2.
In the same manner, we consider the three-dimensional M\"obius
transformation group \textrm{M\"ob}$(\tilde{\mathbb{R}}^3)$
In the three-dimensional case, we define a M\"obius transformation as
the composition of even number of sphere inversions.

\subsection{Inversions in Circles or Spheres}

In the two dimensional case, it is well known that M\"obius transformations on $\tilde{\mathbb{C}}$
are composed of even number of circle-inversions and line-symmetries.
Here, if a circle is centered at $C\in\mathbb{C}$ and it has a radius $R\in\mathbb{R}$ ($R>0$).
Then the formula of the inversion along the circle is given by
$f(z) = \frac{R^2}{~\overline{z -C}~} + C$.
Since $f \circ f$ is the identity map, the inversion map is a homeomorphism on $\tilde{\mathbb{C}}$.
Remark that the inversion is an orientation-reversing conformal map on $\tilde{\mathbb{C}}$.

We often distinguish the inversion from a line symmetry map.
But in this context, by interpreting the line on the complex plane as a circle centered at the infinity, 
we may regard the line symmetry as a kind of the inversion. 
In the sequel, we will call inversion both of circle-inversions and line-symmetries.
In this category, any inversion is orientation-reversing, conformal,
and homeomorphic on $\tilde{\mathbb{C}}$.

Here we will make a short note on the Jacobian of inversion.
If the inversion is a line symmetry, the determinant of the Jacobian is $-1$.
Otherwise if the inversion is a circle-inversion, remarking that a circle inversion
preserve any angles, we obtain that Jacobian mapping is given by the composition of multiplication by a complex constant
and complex conjugate.
Thus the determinant of the Jacobian at a point $P\in\mathbb{C}$ is given by the following formula:
\[ \mathrm{Det}(\mathrm{Jacobian}) = R^4 / distance(P,~C)^4 \]%
Here the Jacobian is a composition of a scaling and a rotation and a complex conjugate, and we obtain the following.
\[ \mathrm{scaling~factor~at~} P  = R^2 / distance(P,~C)^2. \]

In the similar way as the definition of a circle-inversion,
a sphere-inversion can be defined as follows.
Fix a sphere $C$ in $\mathbb{R}^3$.  Let its center be $C$ and its radius be $R$. 
A sphere-inversion $I_C$ is a homeomorphism on $\tilde{\mathbb{R}}^3$ 
determined by a map from a point $P$ to a point $Q = I_C(P)$ such that (i) $Q$ is on the half line $CP$ and 
that (ii) $\overline{CP}\cdot \overline{CQ} = R^2$.
As above, the definition of a circle-inversion can be extended to a plane-symmetry
in the same way as a line-symmetry in the two dimensional case.

\subsection{M\"obius Transformations and the Hyperbolic Geometry}

First we review the definition of a group action. 
Let $G$ be a group and $X$ be a set. A map $G \times X \to X : (g,x) \mapsto g\cdot x$
is called a $G$-action on $X$ if it satisfies the following conditions.
\par \qquad (1) $e \cdot x = x$ for all $x \in X$. (Here, $e$ denotes the unit of $G$.)
\par \qquad (2) $(gh) \cdot x = g \cdot (h\cdot x)$ for all $g, h \in G$ and for all $x \in X$.

In this study, we handle $PSL_2\mathbb{C}$-action on $\hat{\mathbb{C}}$.
Here $PSL_2\mathbb{C}$ is the projective space of $2 \times 2$ complex matrices $A$ with $\mathrm{det}(M)=1$.
M\"obius transformation is defined as a linear fractional transformation
$f(z)=\dfrac{az+b}{cz+d}$ for complex variable $z$ where constants
$a, b, c, d$ are complex numbers and satisfy $ad - bc = 1$.

If we consider a map $\varphi: \dfrac{az+b}{cz+d} \mapsto \begin{pmatrix}a & b \\ c& d \end{pmatrix}$,
from the set of linear fractional transformations to $PSL_2\mathbb{C}$.
This map $\varphi$ is a group isomorphism and it gives an induced action of $PSL_2\mathbb{C}$ on $\hat{\mathbb{C}}$.
In the sequel, we regard the action of linear fractional transformations
and that of $PSL_2\mathbb{C}$ without distinction.
For more details about M\"obius transformation, refer
\cite{MumfordSeriesWright200204}\cite{marden_2016}.

M\"obius transformation is deeply related to the hyperbolic geometry.
When all of the coefficients in $f(z) = \dfrac{az + b}{cz + d}$
are real number, $f(z)$ preserve upper-half plane.
Moreover, $f(z)$ preserves the hyperbolic metric $\dfrac{dz^2}{(\mathrm{Im}z)}$
on the upper-half plane.
Such linear fractional transformations corresponds to an element of $PSL_2\mathbb{R}$.
Thus the $PSL_2\mathbb{R}$-action on the upper half plane is
the isometric transformation group of the hyperbolic plane $\mathbb{H}^2$.

When the coefficients $a, b, c, d$ are complex numbers, 
the action can be thought as an isometric transformation group on the hyperbolic space $\mathbb{H}^3$.
We consider the upperhalf model of the hyperbolic space $\mathbb{H}^3$.
In fact, 
\[\mathbb{H}^3 = \{ (z,t) \in \mathbb{C}\times \mathbb{R}\mid t>0 \} \cup \{\infty\} \]
is the upper half model with the hyperbolic metric $\dfrac{dz^2+dt^2}{t^2}$ and 
$\hat{\mathbb{C}} = \{ (z,0) \mid z \in \mathbb{C}\} \cup \{ \infty\}$
is the set of infinity. The two dimensional M\"obius transformation group acts on $\hat{\mathbb{C}}$
naturally and it is well known that this action can be extend to 
an action on the upper half model as isometric transformations.
This extension is called \textit{\lq Poincar\'e extension\rq}.
As above, researching structures of the M\"obius transformation groups is heavily
related to studying three-dimensional hyperbolic geometry and hence hyperbolic manifolds.
For more details, refer 
\cite{Marden200705outerCircles}\cite{taniguchi_okumura199610invitation}.

In this paper, we also consider the three dimensional M\"obius
transformation on $\hat{\mathbb{R}^3} = \mathbb{R}^3\cup\{\infty\}$.
The definition of the three dimensional M\"obius transformation in the two ways as follows.
One is the simple extension of the two dimensional version.
A M\"obius transformation is defined as a composition of even number of sphere-inversions 
and plane-symmetry.
The other is the representation of the isometry group of the four dimensional hyperbolic space.
To represent three-dimensional M\"obius transformation using matrix,
we consider a quaternion $2 \times 2$ matrix group called
$Sp^k(1,1)$. About this topic, refer
\cite{sakugawa2010limit}\cite{sakugawa2007master}.

Three-dimensional M\"obius transformations are deeply related to
the four-dimantional hyperbolic space. In fact, the upper-half space model of
the four-dimensional hyperbolic space $\mathbb{H}$ is homeomorphic to the 
four-dimensional open ball and its boundary (the infinity point set of the hyperbolic
space) is $\hat{\mathbb{R}^3} = \mathbb{R}^3\cup\{\infty\}$.
Using Poincar\'e extension, a sphere-inversion in $\hat{\mathbb{R}^3}$
is extended to an (orientation-reversing) isometric transformation of $\mathbb{H}^4$.
Thus the composition of even numbers of sphere-inversions corresponds to an orientation-preserving
isometry, and it is known that any isometry of $\mathbb{H}^4$ is given in this way.
In this point of view, three-dimensional M\"obius transformations are
deeply related to the four-dimensional hyperbolic space and four-dimensional hyperbolic manifolds.

\subsection{Classification of M\"obius Transformations}

Excluding the identical mapping, M\"obius transformations
$f(z) = \dfrac{az + b}{cz + d}$ are classified as three types.
They are \textit{Elliptic}, \textit{Parabolic}, and \textit{Loxodromic}.
By the knowledge of linear algebra, the standard form of $PSL_2\mathbb{C}$
is either one of $\begin{pmatrix}1 & a \\ 0 & 1 \end{pmatrix}$ or
$\begin{pmatrix}\lambda & 0 \\ 0 & \lambda^{-1} \end{pmatrix}$.
The former one of $f(z)$ is parabolic transformations.
For parabolic transformation the number of fixed point is one, and
$2 \times 2$ square matrix $X$ satisfy $\mathrm{tr}^2X = 4$.

When the standard form is
$\begin{pmatrix}\lambda & 0 \\ 0 & \lambda^{-1} \end{pmatrix}$
or absolute value of lambda is one, that is  $|\lambda|$ is $1$, 
$f(z)$ is called elliptic transformations.
The properties of elliptic transformation are they have
two fixed points and the transformation gives rotations around fixed
points.
Also, the matrix representation $\mathrm{tr}^2X > 4$ is 
necessary and sufficient conditions.
In addition, the m\"obius transformation having finite order
is elliptic transformation.

When the standard form is
$\begin{pmatrix}\lambda & 0 \\ 0 & \lambda^{-1} \end{pmatrix}$
and complex number $|\lambda| \neq 1$, $f(z)$ is called loxodromic.
If $\lambda$ is real number exclude in $1$, the transformation is called
\textit{hyperbolic} too.
The properties of loxodromic transformation are they have two fixed
points. One of the fixed points is attracting and the other one is repelling.
Also, $\mathrm{tr}^2X < 4$ is necessary and sufficient conditions.

For three-dimensional transformations also have this kind of
classification.
In this case, it is added \textit{simple} or \textit{compound} and
there are six variations of classification like ``simple elliptic'' or
``compound loxodromic''.
For detailed, see chapter 4.3.2.

\subsection{Kleinian Groups}

Kleinian groups are groups originated from name of Felix Klein who is
a mathematician in nineteenth century.
The group $G$ is Kleinian group when it satisfies following two
conditions. 
Firstly, $G$ should be a sub-group of M\"obius transformation group.
Secondly, about a fixed point $x\in\mathbb{H}^3$ in hyperbolic space,
its orbit space $Gx = \{ gx \mid g\in G\}\subset \mathbb{H}^3$
is properly discontinuous.
In this place, sub-set $A$ of hyperbolic space $\mathbb{H}^3$
is properly discontinuous means
for any compact set $K$ of $\mathbb{H}^3$,
there are only finite number of $\gamma$ which is element of $A$
and becomes $\gamma (K) \cap K \neq \phi$.

When a point $x$'s orbit space is properly discontinuous, we take
fundamental domain whose volume is plus.
Considering M\"obius transformation gives isometric transformation of
hyperbolic space, we also consider tiling of hyperbolic space by
Kleinian groups. 

In this paper, we consider M\"obius transformation widely and
think about a mapping which is represented by a composite mapping of
inversion mapping (not limited to even number).
As a term, to distinguish original M\"obius transformation
we call them \textit{extended M\"obius transformation}.
A composite mapping composed of odd number of inversion mapping
is represented by $f(z)=\dfrac{a{\bar{z}}+b}{c{\bar{z}}+d}$ where
$a, b, c, d$ are complex number and $ad-bc = 1$.
However, $\overline{z}$ represents conjugation of complex number.

Extended M\"obius transformation introduced isometric mapping (not
necessarily preserve direction) on three-dimensional hyperbolic space by
poincare expansion.
Properly discontinuous sub-group of extended M\"obius transformation group
is called \textit{extended Kleinian groups}.
Basic property of Kleinian group is described in chapter 2 of \cite{marden_2016}.

\subsection{Limit Set}

Let Kleinian group be $\Gamma$.
We call closed set of all of the limit point \textit{limit set}.
Also, We represent it as $\Lambda(\Gamma)$.
There are some known properties about limit set.
For example,
\begin{itemize}
 \item The orbit space of fixed points of generators becomes limit set.
 \item A point on the limit set transformed by element of the group 
       also a point on the limit set.
 \item Fixed points of generators become limit set.
\end{itemize}

Infinite words $\overline{ab}$ converge to it is called \textit{algebraic limit.}
\textit{geometric limit} is a set of transformed points converging to
the limit.

The property of the limit set is also described in chapter 2.4.1 of
\cite{marden_2016}.

% In this paper, for simplicity, we use
% lower case alphabets such as $a$ and $b$ instead of $f(z)$ and $g(z)$ and
% upper case alphabet such as $A$ and $B$ instead of $f^{-1}(z)$ and $g^{-1}(z)$

% From this point, we assume transformation group $G$ is generated by two
% elements $a$ and $b$. Arbitrary elements of $G$ is represented by
% four alphabet $a$, $b$, $A$, and $B$ and we follow the rule of inverse
%  of $a$ is $A$ and inverse of $b$ is $B$.
% Thus, composite mapping $f(z)\circ g(z) \circ f^{-1}(z)$ simply
% is represented by $abA$.

% Following the rule of the words, we represent circular infinite words as
% bar, that is, $aaaa\cdots$ is represented by $\overline{a}$ and $abABabAB \cdots$ 
% is represented by $\overline{abAB}$.
% These infinite words are not element of $G$. However, when we consider
% the orbit by $G$, we use such notations to express the limit set.