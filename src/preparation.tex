%#BIBTEX biber --bblencoding=utf8 -u -U --output_safechars main
%#!uplatex main.tex

\section{Preparation}

In this section, we introduce some mathematical terms and prerequisites
to understand the IIS algorithm and basic usage of IIS.

\subsection{Terminology}

In this paper, we use terms about Kleinian groups used in Indra's
Pearls \cite{MumfordSeriesWright200204}.
The word \textit{group} represents an algebraic group which is
central concept about group theory.
Algebraic group is a set which has
multiplication and identity element, satysfies associative law, and
whose each element of the group has inverse element.

Also, a transformation group is an algebraic group consists of
transformations about plane or space.
However, in this case, because we assume M\"obius transformation
group, we add infinity $\infty$ to complex plane $\mathbb{C}$ (or three
dimensional space $\mathbb{R}^3$) and apply one-point compactification
to the set.
We consider a group composed of the element of homeomorphic mapping on
the set $\tilde{ \mathbb{C}}$ (or $\tilde{ \mathbb{R}^3 }$.)

We assume that transformations $f(z)$ and $g(z)$ are complex functions
whose parameter is complex number $z$
and homeomorphic mappings on $\hat{\mathbb{C}}$.
What transformation group $G$ is generated by $f(z)$,$g(z)$
is any element of $G$ are represented by some composite mappings of $f(z)$,
$g(z)$, $f^{-1}(z)$, and $g^{-1}(z)$

In this paper, for simplicity, we use
lower case alphabets such as $a$ and $b$ instead of $f(z)$ and $g(z)$ and
upper case alphabet such as $A$ and $B$ instead of $f^{-1}(z)$ and $g^{-1}(z)$

From this point, we assume transformation group $G$ is generated by two
elements $a$ and $b$. Arbitrary elements of $G$ is represented by
four alphabet $a$, $b$, $A$, and $B$ and we follow the rule of inverse
 of $a$ is $A$ and inverse of $b$ is $B$.
Thus, composite mapping $f(z)\circ g(z) \circ f^{-1}(z)$ simply
is represented by $abA$.

Following the rule of the words, we represent circular infinite words as
bar, that is, $aaaa\cdots$ is represented by $\overline{a}$ and $abABabAB \cdots$ 
is represented by $\overline{abAB}$.
These infinite words are not element of $G$. However, when we consider
the orbit by $G$, we use such notations to express the limit set.

\subsection{Inversions in Circles or Spheres}

It is known that M\"obius transformations on $\tilde{\mathbb{C}}$
are composed of even number of inversions in circles.
Here, it is assumed that the inverse mapping about circle centered at
$C\in\mathbb{C}$ and radius $R\in\mathbb{R}$ ($R>0$).
The formula is given by
$f(z) = \frac{R^2}{~\overline{z -C}~} + C$.
According to the definition, inverse mapping is a homeomorphic mapping on
$\tilde{\mathbb{C}}$.

In this context, the circles do not center infinity, but 
by interpreting the line on complex plane as ``the circle whose center is
infinitely far point and radius is infinity'', 
line symmetry transformation (but infinitely far point is
transformed to infinitely far point) on complex plane 
is also inversion in the circle.
Inversion mapping (including line symmetry transformation) does not
preserve direction of complex plane.
Thus compositions of even number of inversion mapping are homeomorphic
mappings preserving direction of complex plane.

Later, we compute Jacobian of inverse mapping.
This is Jacobian matrix as mapping from complex plane to complex plane.
Generally, inverse mapping preserve angles.
From this property, Jacobian mapping is given by multiplication of complex
number.
Concretely, Jacobian is composed of rotations and constant scaling
and absolute value of Jacobian is as follows
$Jacobian = R^2 / distance(P,~C)^2$
where $P$ is a point before applying the inversion.

In the similar manner to circle inversions we can determine inversion
mapping about spherical surface $S^2$ included in $\tilde{\mathbb{R}^3}$.
Here, image of the inversion mapping is determined by center of the sphere and
distance to the center.
Concretely, $f(z) = \frac{R^2}{~\overline{z -C}~} + C$.
Also, a plane $\alpha$ included in $\mathbb{R}^3$ is considered as
sphere whose center and radius are infinity.
In this case, the inversion mapping about the plane $\alpha$ is plane symmetry
transformation about $\alpha$.

\subsection{M\"obius Transformations}

In this study, we handle actions on $PSL_2\mathbb{C}$ of $\hat{\mathbb{C}}$.

M\"obius transformation is defined on $\hat{\mathbb{C}}$ and
complex variable $z$ to linear fractional transformation
$f(z)=\dfrac{ax+b}{cz+d}$ where constants $a, b, c, d$ are complex
number and satisfy $ad - bc = 1$.
Such linear fractional transformation $f(z)$ gives preserving direction of
$\hat{\mathbb{C}}$ and isometric homeomorphic mapping.
As a group acting on $\hat{\mathbb{C}}$, a set of linear fractional
transformation $f(z) = \dfrac{ax + b}{cz + d}$ and $2 \times 2$ matrix
$2\times 2$行列$\begin{pmatrix}a & b \\ c& d \end{pmatrix}$
$PSL_2\mathbb{C}$ and $PSL_2\mathbb{C}$ are same type.
In this place, linear fractional transformation and
complex two-dimensional
projective

Also it is known that any M\"obius transformation can be represented by
even number of circle inversions.
The compositions of M\"obius transformations is also M\"obius
transformations. A set of all of the M\"obius transformations makes
groups. For more details about this topic, refer \cite{marden_2016}.

For a group $G$ mapping $G \times X \to X$ following conditions.
\begin{enumerate}
 \item e $\cdot$ x = x for all x in X. (Here, e denotes the identity element of
       the group G.)
 \item (gh) $\cdot$ x = g $\cdot$ (h$\cdot$ x) for all g, h in G and all x in X.
       (Here, gh denotes the result of applying the group operation of G to the elements g and h.)
\end{enumerate}

This is also called group $G$ acts on space $X$.
In this sense, A set of the linear fractional transformations acts on
$\hat{\mathbb{C}}$

When all of the constant numbers of M\"obius transformations are real number,
that is, $a, b, c, d \in \mathbb{R}$, $f(z)$ preserve upper-half plane.
Moreover, Metric of upper-half plane preserves
$\dfrac{ds^2}{(\mathrm{Im}z)}$
Thus, a set of linear fractional transformations whose all of constants
are real number acts on hyperbolic plane.

When the constants $a, b, c, d$ are complex number, it is thought that
$f(z)$ acts on hyperbolic space.
As the model of hyperbolic space $mathbb{H}^3$, we consider upper half
space model, but we consider three-dimensional space as the coordinates
$(z, t)$, and $z$ is complex coordinates and $t$ is real coordinates.
Upper-half space model is $\{(z,t) \mid t>0\}\cup \{ \infty \}$
universal set.
$\hat{\mathbb{C}} = \{ (z,0) \mid z \in \mathbb{C}\} \cup \{ \infty\}$
is its set of infinite-point. M\"obius transformations are groups acting on
the set of infinite-point. However, it is known that by Poincare
expansion, on hyperbolic plane on hyperbolic space.
M\"obius transformation acts on the set of infinite-point. It is known
that by poincare expansion method.
For the relationship between M\"obius transformations and hyperbolic
space, refer [*]

From the above, researching M\"obius transformation groups is heavily
related to studying three-dimensional hyperbolic geometry and hyperbolic
polyhedra.

Also, in this paper, we also handle a concept extending a M\"obius
transformation to a $\mathbb{R}^3\cup\{\infty\}$.
Concretely we consider even number of inversion mapping about a sphere
and we define three-dimensional M\"obius transformation.
The definition is extending inversions in a circle and M\"obius
transformation .
In that sense, M\"obius transformation should be called two-dimensional
M\"obius transformation, but
simply they are called M\"obius transformation or three-dimensional M\"obius
transformation without confusion.

Three-dimensional M\"obius transformations are deeply related to
four-dimantional hyperbolic space. Actually, we assume four-dimensional
hyperbolic upper half space model, its infinite-point set is
$\mathbb{R}^3\cup\{\infty\}$ and it is known that three-dimensional M\"obius
transformation gives orientation-preserving isometric transformation in
four-dimensional hyperbolic space via poincare expansion.
In this sense studying three-dimensional M\"obius transformation is
deeply related to four-dimensional hyperbolic space or four-dimensional
hyperbolic space and four-dimensional hyperbolic polyhedra.

To represent three-dimensional M\"obius transformation using matrix
there are Quaternion matrix sub-group of two-dimensional square matrix called
$Sp^k(1,1)$. About this topic, refer [*][sakugawa].

\subsection{Classification of M\"obius Transformations}

M\"obius transformations excluding the identical mapping
$f(z) = \dfrac{az + b}{cz + d}$ are classified as three types.
They are \textit{Elliptic}, \textit{Parabolic}, and \textit{Loxodromic}.
By knowledge of linear algebra, the standard form of $PSL_2\mathbb{C}$
is either one of $\begin{pmatrix}1 & a \\ 0 & 1 \end{pmatrix}$ or
$\begin{pmatrix}\lambda & 0 \\ 0 & \lambda^{-1} \end{pmatrix}$.
The former one is parabolic transformations.
For parabolic transformation the number of fixed point is one, and
two-dimensional square matrix $X$ satisfy $\mathrm{tr}^2X = 4$

When the standard form is
$\begin{pmatrix}\lambda & 0 \\ 0 & \lambda^{-1} \end{pmatrix}$
and complex number $|\lambda|$ is $1$, $f(z)$ is called elliptic
transformations. The properties of elliptic transformation are they have
two fixed points and the transformation gives rotations around fixed
points.
Also, matrix representation have $\mathrm{tr}^2X > 4$.
In addition to the m\"obius transformation has finite order,
it is elliptic transformation.

When the standard form is
$\begin{pmatrix}\lambda & 0 \\ 0 & \lambda^{-1} \end{pmatrix}$
and complex number $|\lambda| \neq 1$, $f(z)$ is called loxodromic.
If $\lambda$ is real number exclude in $1$, the transformation is called
hyperbolic.
The properties of loxodromic transformation are they have two fixed
points. One of fixed points is attracting and the other one is repelling.
Also, $\mathrm{tr}^2X < 4$ is necessary and sufficient conditions.

For three-dimensional transformations also have this kind of
classification.
In this case, it is added \textit{simple} or \textit{complex} and
there are six variations of classification.
For detailed definitions of them, see chapter *.

\subsection{Kleinian Groups}

Kleinian groups are groups originated from name of Felix Klein who is
a mathematician in nineteenth century.
The group $G$ is Kleinian group when it satisfies following two
conditions. 
Firstly, $G$ should be sub-group of M\"obius transformation.
Secondly, about $x\in\mathbb{H}^3$ one fixed point 
In this place, hyperbolic space $\mathbb{H}^3$
Its orbit space $Gx = \{ gx \mid g\in G\}\subset \mathbb{H}^3$
is properly discontinuous.
In this place, sub-set $A$ of hyperbolic space $\mathbb{H}^3$
is properly discontinuous means 

When A point $x$'s orbit space is properly discontinuous, we take
fundamental domain whose volume is plus.
Considering M\"obius transformation gives isometric transformation of
hyperbolic space, we also consider tiling of hyperbolic space by
Kleinian groups. 

Properly discontinuous sub-group of extended M\"obius transformation
is called extended Kleinian groups.

In this paper, we consider M\"obius transformation widely and
think about a mapping which is represented by a  composite mapping of
inversion mapping (not limited to even number).
As a term, to distinguish original M\"obius transformation
we call them extended M\"obius transformation.

composite mapping $f(z)=\dfrac{a{\bar{z}}+b}{c{\bar{z}}+d}$ where
$a, b, c, d$ are complex number and $ad-bc = 1$


Also, we call properly discontinuous extended sub-group of M\"obius
 transformation group \textit{extended Kleinian groups}

Basic property of Kleinian group is described in [*]

\subsection{Limit Set}

Let Kleinian group be $\Gamma$.
We call closed set of all of the limit point \textit{limit set}.
Also, We represent it as $\Lambda(\Gamma)$

The orbit space of fixed point of generators becomes limit set.