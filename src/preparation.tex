%#BIBTEX biber --bblencoding=utf8 -u -U --output_safechars main
%#!uplatex main.tex

\section{Preparation}

In this section, we introduce some mathematical terms and prerequisites
to understand the IIS algorithm and basic usage of IIS.

\subsection{Terminology}

In this paper, we use terms about Kleinian groups used in Indra's
Pearls \cite{MumfordSeriesWright200204}.
The word \textit{group} represents algebraic group which is 
central concept about group theory.
This is a set which has
multiplication, satysfies associative law, has identity element, and
each element has inverse element.

Also, a transformation group is an algebraic group consists of
transformations about plane or space.
However, in this case, because we consider M\"obius transformation group,
adding infinity $\infty$ to complex plane $\mathbb{C}$ (or three
dimensional space $\mathbb{R}^3$) and applying one-point compactification.
We consider a group composed of the element of homeomorphic mapping on
the set $\tilde{ \mathbb{C}}$ (or $\tilde{ \mathbb{R}^3 }$.)

We assume that transformations $f(z)$ and $g(z)$ are complex functions
whose parameter is complex number $z$
and homeomorphic mappings on $\hat{\mathbb{C}}$.
What transformation group $G$ is generated by $f(z)$,$g(z)$
is any element of $G$ are represented by some composite mappings of $f(z)$,
$g(z)$, $f^{-1}(z)$, and $g^{-1}(z)$

In this paper, simply we use 
lower case alphabets $a$ and $b$ instead of $f(z)$ and $g(z)$ and
upper case alphabet $A$ and $B$ instead of $f^{-1}(z)$ and $g^{-1}(z)$

From this point, what transformation group $G$ is generated by two
elements $a$ and $b$ arbitrary elements of $G$ is represented by
four alphabet $a$, $b$, $A$, and $B$ and inverse of $a$ is $A$ and
inverse of $b$ is $B$.
Thus, composite mapping $f(z)\circ g(z) \circ f^{-1}(z)$ simply
represent $abA$.

Following the rule of the words, we represent circular infinite words as
bar, that is $aaaa\cdots$ $\bar a$ $abAB \cdots$ $\bar{abAB}$
These infinite words are not element of $G$, but we use 
them to write the limit set when we considerthe orbit of $G$.

%%----
% In this paper, we use the notations about Kleinian groups used in
% Indra's Pearls.
% The word \textit{group} or \textit{transformation group} means a set of
% transformations. 
% For example, we assume there are complex functions $f(z)$, $g(z)$
% where $z$ is a complex number, and their inverse transformations. %%TODO f(z), g(z)についてよりくわしく
% We call the initial elements of the group \textit{generators}.
% We denote transformations in a lowercase alphabet and 
% their inverse in an uppercase alphabet.
% For instance, we write $a$ as $f(z)$, $b$ as $g(z)$, $A$ as $f(z)^{-1}$,
% and $B$ as $g(z)^{-1}$.

% We also denote compositions of transformations as line up letters.
% For example, $f(z) \circ g(z) \circ f(z)^{-1}$ is represented by $abA$.
% We call the compositions of transformations \textit{word}.
% In this way, we can compose an infinite number of transformations;
% We call the set of transformations transformation group.
% An infinitely long word is represented by overline.
% For instance, the word $aaaaa...$ corresponds to $\overline{a}$ and
% $abABabAB...$ corresponds to $\overline{abAB}$.
% The fixed point of the infinite length word corresponds to the action of
% the word.

\subsection{Inversions in Circles or Spheres}

It is known that M\"obius transformations on $\tilde{\mathbb{C}}$
are composed of even number of inversions in circles.
Here, it is assumed that the the inverse map about circle centered at 
$z\in\mathbb{C}$ and radius $R\in\mathbb{R}$ ($R>0$)
is given by
$f(z) = \frac{R^2}{~\overline{z -C}~} + C$
According to the definition, inverse map is a homeomorphic mapping on
$\tilde{\mathbb{C}}$ 

The circle does not center infinity, and we interpret a line on the
complex plane as circle which centered at infinity and radius is
infinity.
The line as \textit{a circle whose center is infinity and radius is
infinity}
the line symmetry transformation
inversion mappings do not preserve direction of the complex plane.
Thus compositions of even number of inversion mapping preserve direction
of complex plane homeomorphic mappings.

Later, we compute Jacobian of inverse mapping.
This is Jacobian matrix as complex plane to complex plane.
Generally, inverse mapping preserve angles.
From this property, Jacobian mapping is multiplication of complex
number.
Concretely, Jacobian is composed of rotations and constant scaling
factor and absolute value is as follows
$Jacobian = R^2 / distance(P,~C)^2$.
where $P$ is a point before applying the inversion

In the similar manner to circle inversions we can determine spherical
surface $S^2$ included in $\tilde{\mathbb{R}^3}$.
We can define inverse map.
Here, image of inverse map is determined by center of the sphere and
distance to the center. Concretely, $f(z) = \frac{R^2}{~\overline{z -C}~} + C$. 
Also, a plane $\alpha$ included in $\mathbb{R}^3$ is considered as
center is infinity and radius is infinity,
We extend surfaces of sphere
In this case, the inverse mapping about plane $\alpha$ is plane symmetry
about $\alpha$.

% It is known that we can construct M\"obius transformations out of a finite
% composition of inversions. Thus, inversion is a kind of primitive
% functions for M\"obius transformations.
% For more details, see the chapter 1 of \textit{Hyperbolic Manifolds}
%  \cite{marden_2016}.

% An inversion in a circle is defined as $f(z) = \frac{R^2}{~\overline{z -C}~} + C$,
% where $z \in \mathbb{C}$ and $C$ and $R$ are center and radius of the circle.
% Note that an inversion in a circle with infinite radius is the same as
% a reflection over a borderline.

% Later, we use Jacobian of the inversions. The equation is as follows.
% $Jacobian = R^2 / distance(P,~C)^2$, where $P$ is a point before
% applying the inversion in the circle and $C$ and $R$ are center and
% radius of the circle.
% Note that the Jacobian of the inversions in a circle with
% infinite radius is $1$.
% Also, sphere inversions and Jacobian of sphere inversions can be
% derived from the same equation to circle inversions.
% An inversion in a sphere with infinite radius is also 1.

\subsection{M\"obius Transformations}

In this study, we handle actions on $PSL_2\mathbb{C}$ of $\hat{\mathbb{C}}$.

M\"obius transformation is defined on $\hat{\mathbb{C}}$ and 
complex variable $z$ to linear fractional transformation 
$f(z)=\dfrac{ax+b}{cz+d}$ where constants $a, b, c, d$ are complex
number and satisfy $ad - bc = 1$.
Such linear fractional transformation $f(z)$ gives preserving direction of
$\hat{\mathbb{C}}$ and isometric homeomorphic mapping.
As a group acting on $\hat{\mathbb{C}}$, a set of linear fractional
transformation $f(z) = \dfrac{ax + b}{cz + d}$ and $2 \times 2$ matrix
$2\times 2$行列$\begin{pmatrix}a & b \\ c& d \end{pmatrix}$
$PSL_2\mathbb{C}$ and $PSL_2\mathbb{C}$ are same type. 
In this place, linear fractional transformation and
complex two-dimensional
projective

Also it is known that any M\"obius transformation can be represented by 
even number of circle inversions.
The compositions of M\"obius transformations is also M\"obius
transformations. A set of all of the M\"obius transformations makes
groups. For more details about this topic, refer[*].
A group $G$


% M\"obius transformations are defined in the extended complex plane,
% $\hat{\mathbb{C}} = \mathbb{C} \cup \{\infty\}$ and expressed as linear
% fractional transformation
% $f(z)=\frac{az + b}{cz + d}$, where $a,~b,~c,~d,~z \in \hat{\mathbb{C}}$.
% The M\"obius transformation which acts on complex plane is represented
% by 2x2 complex number matrix called $PSL(2, \mathbb{C})$.

% 標準型とよぶ.Tの標準形が;z → λz乗法因子が0より大きく,1ではないとき,
% Tは伸縮に共役であり,このときTは双曲型であるという.|λ|=1のときは回転
% 移動に共役であり



% M\"obius transformations are classified into three types as \textit{loxodromic},
% \textit{parabolic}, or \textit{elliptic}.
% Loxodromic transformations have two fixed points and are conjugate to
% scaling by complex numbers except for scaling by unit complex numbers.
% Those whose multiplier is a positive real number
% are also called \textit{hyperbolic} transformations. Parabolic transformations
% have one fixed point and are conjugate to parallel translations.
% Elliptic transformations have two fixed points and are conjugate to rotations.

% On the other hand, the M\"obius transformation which acts on
% three-dimensional space is represented
% by 2x2 quaternion matrix called $Sp^k(1, 1)$.

% However, in this paper, we use compositions of circle or sphere
% inversions to represent M\"obius transformations in many cases.

\subsection{Classification of M\"obius Transformations}

\subsection{Kleinian Groups}

Mathematically, Kleinian groups are discrete sub-group of the M\"obius
transformation groups.  Briefly, discreteness of the group means when we
are given a group (way of tiling), whether we can tile so as not to have
self-intersections and gaps.  If the tiles do not collapsed, the group
is discrete.
%% ポアンカレの多面体定理の話を入れる?
