%#BIBTEX biber --bblencoding=utf8 -u -U --output_safechars main
%#!uplatex main.tex

\section{Visualization of Kleinian Groups}

\subsection{Basic Methods for Visualization}

\begin{figure}[htbp]
 \begin{minipage}[t]{0.5\hsize}
  \center
  \includegraphics[height=1.35in, keepaspectratio]{img/cayleyGraphabcd.png}
  \caption{\textit{Cayley Graph}}
  \label{fig:cayleyGraph}
  \hspace*{\fill}
 \end{minipage}
 \begin{minipage}[t]{0.5\hsize}
  \center
  \includegraphics[height=1.35in, keepaspectratio]{img/preparation/basic/circleOrbit.png}
  \caption{\textit{Orbit of the disks}}
  \label{fig:circOrbit}
  \hspace*{\fill}
 \end{minipage}
\end{figure}

\begin{figure}[htbp]
 \begin{minipage}[t]{0.3333\hsize}
  \center
  \includegraphics[height=1.35in, keepaspectratio]{img/preparation/basic/catCircleOrbit.png}
  \caption{\textit{The orbit of the cat}}
  \label{fig:orbitCat}
  \hspace*{\fill}
 \end{minipage}
 \begin{minipage}[t]{0.3333\hsize}
  \center
  \includegraphics[height=1.35in, keepaspectratio]{img/preparation/basic/hyperbolicTessellation.png}
  \caption{\textit{Hyperbolic Tessellation}}
  \label{fig:hypTiling}
  \hspace*{\fill}
 \end{minipage}
 \begin{minipage}[t]{0.3333\hsize}
  \center
  \includegraphics[height=1.35in, keepaspectratio]{img/preparation/limitSet/limit.png}
  \caption{\textit{Limit set of the Kleinian group}}
  \label{fig:limit}
  \hspace*{\fill}
 \end{minipage}
\end{figure}

\noindent In this sub-section, we introduce basic methods for
visualizing extended Kleinian groups.
We explain a simple example of an extended Kleinian group generated by
four inversions.

First of all, to visualize a group, we consider a \textit{Cayley graph}.
A Cayley graph is defined as follows. Let $G$ be a group, and we fix a
set of generators.
A node of a Cayley graph is an element of $G$, and an edge of a Cayley
graph connects two nodes which are related by right-multiplication of a
generator.
For an easy example, we assume a group generated by four
involutions, which is a map whose square coincides to the identity.
In short, it is the following group.
\[ G=\langle a,b,c,d \mid a^2=b^2=c^2=d^2 = \mathrm{id} \rangle \]

The Cayley graph of the group $G$ is shown in Figure\ref{fig:cayleyGraph}. 
The element of the group is a node. Four edges correspond to 
$a, b, c, \text{and~} d$ are emitted from each node.
Because $a, b, c, \text{and~} d$ are involution, $a=a^{-1}$, and
an edge corresponds to an inverse map is not needed.
In Figure \ref{fig:cayleyGraph}, we put the identity element $\mathrm{id}$
to the center and four edges are stretching symmetrically.
In a Cayley graph, the graph distance from the identity to a node gives 
a word representing the node.

We can find that the number of words of the same length grows multiplied by three, 
that is to say words with length one is four, the number of words with length two are
twelve, the number of words with length three are thirty-six, and the
number of words with length $n$ are $4 \cdot 3^{n-1}$

The visualized image is Figure \ref{fig:circOrbit}. In this image,
we draw four circles on the plane, and we call their inversion
$a, b, c,$ and $ d$ respectively. Four circles come in touch but do not cross,
and hence there is no relational expression between $a$, $b$, $c$, and $d$.
Thus, the extended M\"obius transformation group generated by these four inversions
is group isomorphic to above $G$.
In Figure \ref{fig:orbitCat}, the cat is drawn in the center is moved by
inversions, and ``the orbit space of the cat'' is visualized.
We can find that the orbit space of the cat corresponds to nodes of
the graph.

In order to draw the orbit space of the cat, there is breadth first search
algorithm.
In the following, we explain breadth first search algorithm.
First of all, we draw the cat in the center, and
next, we draw transformed cats by the words whose length is one,
and we draw transformed cats by the word whose length is two.
We continue iterating these processes.
This is breadth first search algorithm.
The tessellation shown in Figure \ref{fig:hypTiling} is also drawn by
breadth first search algorithm.

Visualization by this algorithm is easy for us to implement and understand.
We can roughly understand the actions of the groups.
However, this method has a fault because the computational complexity is 
easy to increase and it takes too much time.

Also, there is another method of traversing the graph.
It is called depth first search.
Using this algorithm, we can draw only the limit set directly.
See Figure \ref{fig:limit}. This shows an only the limit set of a Kleinian
group.
However, we do not use this algorithm in this paper.
For more details of the above methods, read \cite{MumfordSeriesWright200204}.

\subsection{Iterated Inversion System (IIS)}

\begin{figure}[htbp]
 \begin{minipage}[t]{0.16\hsize}
  \center
  \includegraphics[width=1in, height=1in, keepaspectratio]{./img/preparation/orbit/level0c.pdf}
  \subcaption{}
  \label{fig:level0}
 \end{minipage}
 \begin{minipage}[t]{0.16\hsize}
  \center
  \includegraphics[width=1in, height=1in, keepaspectratio]{./img/preparation/orbit/level0invc.pdf}
  \subcaption{}
   \label{fig:level0inv}
 \end{minipage}
 \begin{minipage}[t]{0.16\hsize}
  \center
  \includegraphics[width=1in, height=1in, keepaspectratio]{./img/preparation/orbit/level1c.pdf}
  \subcaption{}
   \label{fig:level1}
 \end{minipage}
 \begin{minipage}[t]{0.16\hsize}
  \center
  \includegraphics[width=1in, height=1in,
  keepaspectratio]{./img/preparation/orbit/level1invc.pdf}
  \subcaption{}
  \label{fig:level1inv}
 \end{minipage}
 \begin{minipage}[t]{0.16\hsize}
  \center
  \includegraphics[width=1in, height=1in, keepaspectratio]{./img/preparation/orbit/level2c.pdf}
  \subcaption{}
  \label{fig:level2}
 \end{minipage}
 \begin{minipage}[t]{0.16\hsize}
  \center
  \includegraphics[width=1in, height=1in, keepaspectratio]{img/preparation/orbit/levelMaxc.pdf}
  \subcaption{}
  \label{fig:levelMax}
 \end{minipage}
 \caption{\textit{The process of rendering the orbit of disks}}
 \label{fig:schottkyProcess}
\end{figure}

\noindent To solve the problems we discussed in the previous sub-section,
we focus circle inversions and invent an efficient algorithm to
visualize the circle inversion fractals shown in Figure \ref{fig:schottkyProcess}\subref{fig:levelMax}.
The algorithm is called \textit{Iterated Inversion System (IIS.)}
It can visualize not only two-dimensional circle inversion fractals but
also three-dimensional sphere inversion fractals.

The fractal in Figure \ref{fig:schottkyProcess}\subref{fig:levelMax} shows the orbit of the first
four circles in Figure \ref{fig:schottkyProcess}\subref{fig:level0}.
It is also a Kleinian group composed of four circle inversions.
Especially, it is called a \textit{fuchsian group} when the limit set
shapes into a perfect circle.
On the other hand, the group called \textit{quasi-fuchsian group}
when the limit set shapes into a closed curve excluding a perfect circle.
The process of generation of circle inversion fractals is
as follows.

\begin{enumerate}
 \item We need some disjoint disks to obtain circle inversion fractals.
       For example, we assume there are four orange disks as shown in
       Figure \ref{fig:schottkyProcess}\subref{fig:level0}. We call orange disks \textit{initial
       disks} and their boundary \textit{initial circles}.
 \item First of all, we focus on the white circle in Figure
       \ref{fig:schottkyProcess}\subref{fig:level0inv}. The inversion in the white circle moves the
       other three disks into the interior of the white circle.
 \item After we apply each inversion in the initial circle to the outer disks,
       we obtain twelve small disks. They are shown in Figure \ref{fig:schottkyProcess}\subref{fig:level1}.
 \item Next, we invert the twelve small disks in the initial circles.
       The inversion in the white circle moves the outer nine small disks
       into the interior of the white circle as shown in Figure \ref{fig:schottkyProcess}\subref{fig:level1inv}.
       Each inversion in the initial circle generates smaller disks, and we
       obtain Figure \ref{fig:schottkyProcess}\subref{fig:level2}.
 \item We continue iterating these processes, that is, we continue
       applying each inversion in the initial circle to resulting
       smaller disks.
       Finally, we get Figure \ref{fig:schottkyProcess}\subref{fig:levelMax}.
\end{enumerate}

\subsubsection{Two Dimensional IIS}

\begin{figure}[htbp]
  \center
  \includegraphics[height=1.35in, keepaspectratio]{img/preparation/orbIIS.png}
  \caption{\textit{Orbit of blue point by IIS}}
  \label{fig:iisOrbit}
 \hspace*{\fill}
\end{figure}

 \begin{algorithm}
  \caption{Iterated Inversion System (IIS)}
  \label{arg:iis2d}
  \begin{algorithmic}
   \REQUIRE count $= 0$ and coordinates $=$ position determined by
   pixel
   \FOR{$i=0$ to MAX\_INVERSION}
   \STATE inOutside $\leftarrow$ \TRUE
   \FOR{ each Map $G$ in circles }
   \IF{$G$ is available to coordinates}
   \STATE coordinates $\leftarrow$ $G(\text{coordinates})$
   \STATE INCREMENT count
   \STATE inOutside $\leftarrow$ \FALSE
   \ENDIF
   \ENDFOR
   \IF {inOutside}
   \STATE BREAK for
   \ENDIF
   \ENDFOR
   \STATE RETURN count
  \end{algorithmic}
 \end{algorithm}

\noindent IIS computes the depth of the circles point by point.
Thus, we can perform parallel processing and render the images efficiently.
All of the images in this paper are rendered using
\textit{OpenGL Shading Language (GLSL)}.

IIS can be applied to each point on the plane and computes nesting depth of
the disk which contains the point.
The process of the algorithm is as follows.
First of all, if the point is contained in initial disks, we invert the
point in the boundary circle of the disk.
We continue applying inversions until the transformed point is in the
outside of the initial disks.
Figure \ref{fig:iisOrbit} shows the orbit of the blue point transformed by
iterations of inversions.

Furthermore, a point actually at the limit set never reaches the
outside. So, we have to determine the maximum number of iterations in
advance to prevent the algorithm from running indefinitely.
The points except for the limit set are guaranteed that they
are transformed to outside because inversions are involution.

Pseudo-code of IIS is shown in Algorithm \ref{arg:iis2d}.
Later in section 4.3, we will introduce generators other than simple inversions.
Thus, we consider a map $\gamma : \mathbb{C} \to G$ such that $\gamma(p)$ is the unit for a point $p$ in the
\textit{fundamental domain} and that $G \to \gamma(p)$ is a composition of inversions for other
points.
The fundamental domain is the terminal area of transformations. 
In the above case, the black area consists of the outside and the
boundary of all circles is the fundamental domain.

\subsubsection{Three Dimensional Extension}

\begin{figure}[htbp]
 \begin{minipage}[t]{0.5\hsize}
  \center
  \includegraphics[height=1.35in, keepaspectratio]{img/preparation/3dExtension/3dKissingGenerator.png}
  \subcaption{\textit{Generator}}
  \label{fig:simpleGen}
  \hspace*{\fill}
 \end{minipage}
 \begin{minipage}[t]{0.5\hsize}
  \center
  \includegraphics[height=1.35in, keepaspectratio]{img/preparation/3dExtension/3dOrbit.png}
  \subcaption{\textit{The orbit of spheres}}
  \label{fig:simpleOrb}
  \hspace*{\fill}
 \end{minipage}
 \caption{\textit{The orbit of the sphere inversion fractal}}
 \label{fig:simpleGenOrb}
\end{figure}

\begin{figure}[htbp]
 \begin{minipage}[t]{0.5\hsize}
  \center
  \includegraphics[height=1.35in, keepaspectratio]{img/visualization/sphereTracing1.png}
  \subcaption{\textit{}}
  \label{fig:st1}
  \hspace*{\fill}
 \end{minipage}
 \begin{minipage}[t]{0.5\hsize}
  \center
  \includegraphics[height=1.35in, keepaspectratio]{img/visualization/sphereTracing2.png}
  \subcaption{\textit{}}
  \label{fig:st2}
  \hspace*{\fill}
 \end{minipage}
 \begin{minipage}[t]{0.5\hsize}
  \center
  \includegraphics[height=1.35in, keepaspectratio]{img/visualization/sphereTracing3.png}
  \subcaption{\textit{}}
  \label{fig:st3}
  \hspace*{\fill}
 \end{minipage}
 \begin{minipage}[t]{0.5\hsize}
  \center
  \includegraphics[height=1.35in, keepaspectratio]{img/visualization/sphereTracing4.png}
  \subcaption{\textit{}}
  \label{fig:st4}
  \hspace*{\fill}
 \end{minipage}
 \caption{\textit{Sphere tracing}}
 \label{fig:sphereTracing}
\end{figure}

In a similar manner to the two-dimensional algorithm,
we extend the IIS to visualize three-dimensional Kleinian fractals.
We extend the IIS algorithm by replacing circle inversions to sphere inversions easily, 
and we compute the nesting depth of the sphere voxel by voxel.

We use \textit{ray tracing} to visualize three-dimensional objects.
Ray tracing computes an intersection between a ray and objects
algebraically.
When we consider three-dimensional sphere inversion fractals in a similar
manner to circle inversion fractals, we can obtain an image of an
infinite number of nesting spheres in a sphere.
There are two ways to render the nesting spheres. First one is to draw
the nesting spheres as transparent spheres.
Second one is \textit{volume rendering}.
However, it is difficult to render these two way of imaging efficiently, and
visualized images are not helpful to study Kleinian groups.

Therefore we render the orbit of the sphere in a similar way to the
two-dimensional circle inversion fractals by using IIS.
See Figure \ref{fig:simpleGenOrb}\subref{fig:simpleGen}. It shows
white six inversion spheres and a green seed sphere.
Figure \ref{fig:simpleGenOrb}\subref{fig:simpleOrb} shows the orbit of
the green sphere transformed by inversions in white spheres.

We use \textit{sphere tracing} \cite{hart1996sphere} to render three-dimensional
fractals and orbit of the seed sphere.
Sphere Tracing is one of the algorithms to render implicit surfaces using
ray tracing.
In the following paragraphs, we introduce ray tracing and
sphere tracing.

In the first place, we work a ray as something like a vector.
We set the origin of the ray to the position of the camera
and direction of the ray to the direction to each pixel on the screen
from the camera. Each pixel is colored according to the
first object the ray hits. 

In the regular ray tracing, we calculate the intersections between the
ray and the object algebraically, but
it is impracticable to compute the intersections to fractal objects.
As an alternating method, in sphere tracing, we make the tip of the
ray marching along the direction of the ray step by step. 
To check how far the tip of the ray is from the objects, we need a
\textit{distance function}.
The distance function is a function which returns the minimum distance
between a given point and the object.
For example, a distance function of a sphere can be obtained easily.
Let $P$ be a tip of a ray, let $C$ and $R$ be the center and the radius
of the sphere respectively.
A distance function of sphere $f(P)$ is given by $f(P) = distance(P, C) - R$.
If there are many spheres, we use the minimum of the distance functions
of the spheres.

Figure \ref{fig:sphereTracing} shows a sphere tracing algorithm.
The yellow arrow is a ray and orange disks are the visible objects.
First of all, in Figure \ref{fig:sphereTracing}\subref{fig:st1},
we use the distance function and find the minimum distance between
the tip of the ray and the disks.
Next, we put the tip forward and apply distance function to the position
of the ray as in Figure \ref{fig:sphereTracing}\subref{fig:st2}.
Again, we apply the distance function and make the tip marching, and we obtain
Figure \ref{fig:sphereTracing}\subref{fig:st3}.
We continue iterating this processes until the value of the distance
function is less than a small positive threshold as shown in Figure
\ref{fig:sphereTracing}\subref{fig:st4} or sufficiently large.

However, in regard to fractal rendering, it is difficult to
get an actual distance from a given point to the object.
So, we use a lower estimated distance
as a return value of the distance function. The technique to approximate
a distance is called \textit{distance estimation}.
For more details about fractal rendering and distance estimation, see also the blog
post\footnote{Mikael H Christensen, Distance Estimated 3D Fractals (Part I):\\ \quad\quad
\url{http://blog.hvidtfeldts.net/index.php/2011/06/distance-estimated-3d-fractals-part-i/}}
by Christensen. 

\begin{figure}[htbp]
 \begin{minipage}[t]{0.5\hsize}
  \center
  \includegraphics[height=1.35in, keepaspectratio]{img/visualization/translate1.png}
  \subcaption{\textit{}}
  \label{fig:modulo1}
  \hspace*{\fill}
 \end{minipage}
 \begin{minipage}[t]{0.5\hsize}
  \center
  \includegraphics[height=1.35in, keepaspectratio]{img/visualization/translate2.png}
  \subcaption{\textit{}}
  \label{fig:modulo2}
  \hspace*{\fill}
 \end{minipage}
 \caption{\textit{Fold space by modulo operator}}
 \label{fig:moduloAll}
\end{figure}

\begin{figure}[htbp]
 \begin{minipage}[t]{0.5\hsize}
  \center
  \includegraphics[height=1.35in, keepaspectratio]{img/visualization/scaling1.png}
  \subcaption{\textit{}}
  \label{fig:iisScale1}
  \hspace*{\fill}
 \end{minipage}
 \begin{minipage}[t]{0.5\hsize}
  \center
  \includegraphics[height=1.35in, keepaspectratio]{img/visualization/scaling2.png}
  \subcaption{\textit{}}
  \label{fig:iisScale2}
  \hspace*{\fill}
 \end{minipage}
 \caption{\textit{Fold space by scaling}}
 \label{fig:iisScaleAll}
\end{figure}

Before we introduce details of distance estimation, we show
a well-known technique to render many objects in sphere tracing.
Instead of preparing distance functions for many objects lined up in a
row, we can use modulo operator
to get the distance to the lined up objects.

See Figure \ref{fig:moduloAll}\subref{fig:modulo1}. 
We assume there are the blue disk, the
yellow ray, and dotted circle and lines.
Now, we want to draw all of the dotted circles.
So, we want the minimum distance between the tip of the ray and dotted circles.
If we only know the position of the tip of the ray and the blue disk,
then
how can we get the minimum distance?
Thus, we assume the nearest circle to the ray is in the same dotted region, and
we fold up the regions using the modulo operator.
We can measure the distance to blue disk as in Figure \ref{fig:moduloAll}\subref{fig:modulo2}.
Finally, we can obtain the minimum distance to line upped disks.

Next, we consider scaling example.
See Figure \ref{fig:iisScaleAll}\subref{fig:iisScale1}.
There are a yellow ray and a blue disk, and orbit of scaling as
dotted blue circles.
The green dotted lines show scaling regions.
We want to get the minimum distance between dotted blue circles and the
tip of the ray.
If we only know the coordinates of the original blue
disk and tip of the ray, then how can we get the minimum distance?
We assume that the nearest sphere to the tip of the ray is in
same regions. 
Thus, we scale the tip of the ray to the same region to the blue disk
but, the distance between scaled ray and disk is also scaled.
So, we correct the distance by the ratio determined by Jacobian
(sometimes referred to as the Jacobian determinant) of scaling transformation.
See Figure \ref{fig:iisScaleAll}\subref{fig:iisScale2}.
If the scaling of the circle is $f(z) = 2z$, we divide the distance by
$2^{-3}$ because the tip of the ray is moved to the three regions away
from the original.

\begin{figure}[htbp]
 \begin{minipage}[t]{0.5\hsize}
  \center
  \includegraphics[height=1.35in, keepaspectratio]{img/preparation/slice.png}
  \caption{\textit{XY-slice image of Figure \ref{fig:simpleGenOrb}\subref{fig:simpleOrb}}}
  \label{fig:slice2d}
  \hspace*{\fill}
 \end{minipage}
 \begin{minipage}[t]{0.5\hsize}
  \center
  \includegraphics[height=1.35in, keepaspectratio]{img/preparation/artifact.pdf}
  \caption{\textit{artifact}}
  \label{fig:3dartifact}
  \hspace*{\fill}
 \end{minipage}
\end{figure}

\begin{algorithm}
 \caption{Distance Function}
 \label{iis3d}
 \begin{algorithmic}
  \REQUIRE count $= 0$, $d$ = MAX\_DISTANCE, $dr = 1.0$, and coordinates
  $=$ tip of the ray
  \FOR{$i=0$ to MAX\_INVERSION}
  \STATE inFundamentalDomain $\leftarrow$ \TRUE
  \FOR{ each Map $G$ in Maps}
  \IF{$G$ is available to coordinates}
  \STATE $dr \leftarrow dr * $ (Jacobian of $G(\text{coordinates})$)
  \STATE coordinates $\leftarrow$ $G(\text{coordinates})$
  \STATE INCREMENT count
  \STATE inFundamentalDomain $\leftarrow$ \FALSE
  \ENDIF
  \ENDFOR
  \IF {inFundamentalDomain}
  \STATE BREAK for
  \ENDIF
  \ENDFOR
  \FOR{ each BaseSphere $S$ in BaseSpheres}
  \STATE $d \leftarrow$ min($d$, scalingFactor $*$ (distance(coordinates, $S$.center) $-$
  $S$.radius) $/$ (absolute value of $dr$))
  \ENDFOR
  \RETURN $d$
 \end{algorithmic}
\end{algorithm}

In a similar manner to scaling, we can render sphere inversion
fractals.
For the sake of simplicity, we consider a slice image of Figure
\ref{fig:simpleGenOrb}.
See Figure \ref{fig:slice2d}. This image shows the XY-slice of the orbit
of spheres.
Orange disks in the background are slices of the orbit of the initial inversion spheres.
Slices of the orbit of the base sphere are colored in
the same color as the orbit shown in Figure \ref{fig:simpleGenOrb}\subref{fig:simpleOrb}.
In Figure \ref{fig:slice2d},
$C$ is the white circle, the boundary of the initial inversion sphere.
$S1$ is the base sphere and the inversion of $S2$ in the circle $C$. 
The white point $P1$ is the inversion of $P2$ in the circle $C$.

Now we assume that the tip of the ray is at $P2$.
Let's calculate the minimum distance between
$P2$ and the orbit of base spheres.
The nearest sphere to $P2$ is $S2$.
So, we have to calculate the distance between $P2$ and $S2$.
We call the distance $d$.
However, we do not know the center and radius of $S2$.
So, we calculate $d$ from a distance between $S1$ and $P1$.
Inversions in spheres and M\"obius transformations do not preserve
Euclidean distance.
Thus we use the Jacobian to estimate the distance.
We accumulate the Jacobian of inversions by multiplying the Jacobian for
every inversion.

Finally, we divide the distance between the base sphere and the point on
the fundamental domain by the accumulated Jacobian, and we can get the
approximated distance between the tip of the ray and the nearest sphere.
For the above case, we get an inequality $d \geq distance(P1, S1)/Jacobian$.
The formula gives a lower bound for spheres.
For more details on the derivation of this estimation formula, see the
blog post\footnote{Inigo Quilez, distance estimation:
\url{http://www.iquilezles.org/www/articles/distance/distance.htm}}
by Inigo Quilez.

We have one more thing to consider because the above calculation is
a rough estimate.
For example, if a given point is in the outer area of the orbit, the
distance function returns unintentionally large distance, and the ray
can pass through the real objects. This causes artifact shown in Figure
\ref{fig:3dartifact}. The fore part of the fractals is not rendered.
In order to avoid this kind of trouble, we shrink the length of
the estimated distance.
Shrinking the length of the distance increases the number of steps of
sphere tracing, but we can
eventually obtain the intersection of the ray and the spheres.
The scaling factor is determined experimentally according to the size of
the spheres.

The generalized pseudo-code for a distance function is in Algorithm \ref{iis3d}. 

\subsection{Related Works}

Before IIS is developed, there is an algorithm to draw a fractal called
\textit{pseudo-kleinian}, and there are visualization techniques and
variations related to pseudo-kleinian.
The algorithms are fast same as IIS, but pseudo-kleinian is not a Kleinian
group.
Pseudo-kleinian uses M\"obius transformation. So, it is similar to IIS in
terms of to draw three-dimensional fractals efficiently.
Thus, there is a relationship between pseudo-kleinian and Kleinian groups.
However, pseudo-kleinian differs from Kleinian groups which the
mathematicians study.

\begin{figure}[htbp]
 \begin{minipage}[t]{0.5\hsize}
  \center
  \includegraphics[height=1.35in, keepaspectratio]{img/preparation/related/josklein.png}
  \caption{\textit{The limit set of the Kleinian groups with Maskit parametrization.}}
  \label{fig:jos}
  \hspace*{\fill}
 \end{minipage}
 \begin{minipage}[t]{0.5\hsize}
  \center
  \includegraphics[height=1.35in, keepaspectratio]{img/preparation/related/joskleinInv.png}
  \caption{\textit{The limit set applied inversions of the circles.}}
  \label{fig:josInv}
  \hspace*{\fill}
 \end{minipage}
\end{figure}

Aaron Montag uses texture based approach to visualize a limit set of the
Kleinian groups \cite{Montag2014hyperbolicIFS}.
We prepare initial seed circle in the texture.
Next, we apply generators of the group to each pixel.
If the transformed pixel is on the seed circle or filled pixel, we fill the original pixel.
We continue iterating this process; we obtain an image of the limit set of
the group.
This algorithm needs high-resolution texture, and it is difficult to
extend this algorithm to three-dimension.

Jos Leys invented efficient algorithm to draw a Kleinian group with Maskit
parametrization shown in Figure
\ref{fig:jos}\footnote{\url{http://www.josleys.com/article_show.php?id=221}}. 
The pink points are control points of the fractal.
Although IIS uses circle or sphere inversions, Jos Leys used
M\"obius transformations algebraically.
He observes the orbit of the Maskit parametrization group and
successfully discovers the algorithm to visualize Maskit groups.
We can apply circle inversions to his figure, and we obtain Figure \ref{fig:josInv}.

Martin von Gagern and J\"urgen Richter-Gebert introduced an algorithm
called \textit{Reverse Pixel Lookup}
\cite{journals/combinatorics/GagernR09} to render two-dimensional
hyperbolic tiling.
The algorithm is similar to ours, but
we explore the method not only tiling but also other varieties of
images, for instance, three-dimensional objects.
