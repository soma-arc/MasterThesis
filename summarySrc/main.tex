%#BIBTEX biber --bblencoding=utf8 -u -U --output_safechars main
\documentclass[uplatex, dvipdfmx]{article}
\usepackage{amsmath, amsthm, amssymb}
\usepackage{ascmac,here,txfonts,txfonts}
\usepackage[margin=2.5cm]{geometry}
\usepackage{tikz}
\usetikzlibrary{arrows}
\usepackage{graphicx}
\usepackage{algorithm}
\usepackage{algorithmic}
\usepackage{color}
\usepackage{xcolor}
\usepackage[subrefformat=parens]{subcaption}
\usepackage{hyperref}
\usepackage[backend=biber, style=numeric, sorting=none]{biblatex}
\usepackage[format=hang]{caption}
\usepackage[format=hang, subrefformat=parens]{subcaption}

\captionsetup{compatibility=false}

\addbibresource{./references.bib}

\setlength{\parindent}{0.3in}

\title{修士学位請求論文要旨\\
 Iterated Inversion System: An Efficient Algorithm to Visualize Kleinian Groups Based on Inversions }
\author{中村 建斗\\
先端数理科学研究科 先端メディアサイエンス専攻\\
}

\date{}

\pagestyle{plain}

\begin{document}

\maketitle
\pagestyle{plain}
\newpage

In this paper, we introduce an efficient algorithm to visualize Kleinian
groups.
Kleinian groups theory is one of the fields of mathematics.
It is suited to visualization and computer experiments.
Summary of each chapter is as follows.

%% 1
The first chapter is introduction. 
We show background of the research.
%% 2
The second chapter is preparation.
we introduce some mathematical terms like inversion in circle, 
M\"obius transformations, or Kleinian groups.
%% 3
The third chapter introduce visualization of Kleinian groups.
In the chapter 3.1, basic method of Kleinian groups.
It has some defective to visualize Kleinian groups.
To solve this problem,
After chapter 3.2, we introduce the Iterated Inversion System (IIS).
It can be use two and three dimensional.
In chapter 3.3, we show the algorithm to draw three dimensional
obbjects using IIS.
We use \textit{sphere tracing}, a kind of ray tracing algorithm.
To render three dimantional objects, Jacobian is important.

%% 4
The fourth chapter is applications.
We introduce four application of IIS.

First one is how to render the interior or exterior of the
two-dimensional circle inversion fractals.

Secondly, the method to draw edge of circles.

Thirdly
the method to generate two or three dimensional Kleinian groups with
circle or sphere inversion.

Sphairahedron fractals. Sphairahedron is polyhedron with spherical
faces. We can make a tiling pattern of the sphairahedron.
In many cases, the boundary of the tiling converge to a three
dimensional fractal. 

%% 5

Finally, chapter 5 is conclusion.

\end{document}
